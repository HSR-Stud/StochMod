\section{Ereignisse und ihre Wahrscheinlichkeit}
%\renewcommand{\baselinestretch}{1.25}\normalsize
	\subsection{Kombinatorik}
		\begin{minipage}{13.5cm}
		\begin{tabular}{| p{5.5cm} | c | c |}
			\hline
			Art der Auswahl bzw. Zusammenstellung von $k$ aus $n$ Elementen
			& \multicolumn{2}{c|}{Anzahl der Möglichkeiten}\\
 			& ohne Wiederholungen		& mit Wiederholungen\\
 			& $(k\leq n)$ 				& $(k\leq n)$ \\
 			\hline
 			Permutationen & $P_n=n!(n=k)$ &
 			$P_n^{(k)}=\frac{n!}{k!}$ \\ & &\\
 			Kombinationen & $C_n^{(k)}=\binom n k$ &
 			$C_n^{(k)}=\binom{n+k-1} k$\\
 			& &\\
 			Variationen & $V_n^{(k)}=k!\binom n k$ & $V_n^{(k)}=n^k$\\
 			\hline
		\end{tabular}
		\end{minipage}
		\begin{minipage}{5cm}
		$\binom n k$ mit TR: \\
		\texttt{nCr(n,k)} \hspace{19.5mm}En\\
		\texttt{Kombinat(n,k)} \hspace{9.3mm} De
		\end{minipage}
		\begin{list}{$\bullet$}{\setlength{\itemsep}{0cm} \setlength{\parsep}{0cm} \setlength{\topsep}{0.1cm}} 
         	\item \textbf{Permutationen}: Gegeben seien $n$ verschiedene Objekte. Dann gibt es $n!$
         	verschiedene Reihenfolgen in denen man diese Objekte anordnen
         	kann. \\
         	z.B.: $x,y,z;\quad x,z,y;\quad z,y,x;\ldots$
         % \item Permutation nennt man eine Anordnung von $n$ Elementen in einer Bestimmten
		%		Reihenfolge
		 	\item \textbf{Kombination}: Gegeben seien $n$ verschiedene Objekte. Dann gibt es $\binom n k$
		 	Möglichkeiten, daraus $k$ Objekte auszuwählen, wenn es nicht auf die Reihenfolge
		 	ankommt. \\
		 	z.B.: Wie viele verschiedene Möglichkeiten hat man beim Lotto, 6 Zahlen aus 49
		 	auszuwählen?
		 % \item Kombination nennt man eine Auswahl von $k$ Elementen aus $n$ Elementen
		 % 		ohne Beachtung der Reihenfolge
		  \item \textbf{Variation} nennt man eine Auswahl von $k$ Elementen aus $n$
		  		verschiedenen Elementen unter Beachtung der Reihenfolge
        \end{list}
        

\vspace{5mm}
	\begin{minipage}{6.8cm}
	\subsection{Wahrscheinlichkeit}
		\begin{tabular}{ll}
			Wertebereich:
			& ${0}\le{P(A)}\le{1}$\\ \\
			Sicheres Ereignis:
			& $P(\Omega)=1$\\ \\
			unmögliches Ereignis:
			& $P(\emptyset)=0$
		\end{tabular}
	\end{minipage}
		\begin{minipage}{11.2cm}
		\subsubsection{Rechenregeln}
			\begin{tabular}{ll}
				komplementär Ereignis:
				&$P(\bar{A})=P({\Omega}\setminus{A})=1-P(A)$\\ \\
				Differenz der Ereignisse A und B:
				&$P({A}\setminus{B})=P(A)-P({A}\cap{B})$\\ \\
				Vereinigung zweier Ereignisse:
				&$P({A}\cup{B})=P(A)+P(B)-P({A}\cap{B})$
			\end{tabular}
		\end{minipage}
\vspace{1mm}

	
	\subsection{Laplace-Ereignisse}
    	In einem endlichen Wahrscheinlichkeitsraum $\Omega$ haben alle
    	Elementarereignisse die gleiche Wahrscheinlichkeit.
    	\begin{center}
    	$P(A)=\dfrac{\left| A\right|}{\left|\Omega\right|}$
    	\end{center}




	\subsection{Unabhängige Ereignise \skript{15}/\skript{22}}
		Unabhängige Ereignisse $A$ und $B$ liegen vor, wenn:\\
    	\hspace*{8mm} $P(A\mid B)=P(A)$ \hspace{4mm} und \hspace{4mm}
    	$P(B\mid A)=P(B)$\\
    	erfüllt ist. Für sie gilt\\
    	\hspace*{8mm} $P(A\cap B)=P(A)P(B)$\\
    	Die Tatsache, dass A eingetreten ist, hat keinen Einfluss auf die 
		Wahrscheinlichkeit von B.\vspace{1mm}


	\subsection{Bedingte Wahrscheinlichkeit \skript{18}}
		Die Wahrscheinlichkeit für das Eintreten des Ereignisses $A$ unter der
		Bedingung, dass das Ereignis $B$ bereits eingetreten ist.
		\begin{center}
		$P(A\mid B)= \dfrac{P(A\cap B)}{P(B)}=\underbrace{\frac{P(A)\cdot
		P(B)}{P(B)}=P(A)}_{\text{nur wenn unabhängig}}$ 
		\end{center}

	\subsection{Satz von Bayes \skript{20}}
		\begin{tabular}{ll}
		$P(B\mid A)=P(A\mid B) \cdot\dfrac{P(B)}{P(A)}$\vspace{1mm}
		\end{tabular}


	\subsection{Totale Wahrscheinlichkeit \skript{19}}
		\begin{tabular}{ll}
        $P(A)=\sum\limits_{i=1}^N P(A\mid G_i)\cdot P(G_i)$
        \end{tabular}

